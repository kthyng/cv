% Jason R. Blevins - Curriculum Vitae
%
% Copyright (C) 2004-2010 Jason R. Blevins
% http://jblevins.org/projects/cv-template/
%
% You may use use this document as a template to create your own CV
% and you may redistribute the source code freely. No attribution is
% required in any resulting documents. I do ask that you please leave
% this notice and the above URL in the source code if you choose to
% redistribute this file.

\documentclass[10pt,letterpaper]{article}

\usepackage{hyperref}
\usepackage{geometry}
\usepackage[T1]{fontenc}

\newcommand{\kmt}{\textbf{K. M. Thyng}}
\newcommand{\tkm}{\textbf{Thyng, K. M.}}
\newcommand{\kt}{\textbf{K. Thyng}}
\newcommand{\inv}{\textbf{\textit{(invited)}}}

% Comment the following line to use the default Computer Modern font
% instead of the Palatino font provided by the mathpazo package.
% Remove the 'osf' bit if you don't like the old style figures.
\usepackage[sc,osf]{mathpazo}

% In practice, I use the following font packages instead of mathpazo.
% beramono provides a nice fixed-width font. xagaramon uses the
% (commercial) Adobe Garamond font.
%\usepackage[scaled=0.75]{beramono}
%\usepackage[osf]{xagaramon}

% Set your name here
\def\name{Kristen M. Thyng}

% The following metadata will show up in the PDF properties
\hypersetup{
  colorlinks = true,
  urlcolor = black,
  pdfauthor = {\name},
  pdfkeywords = {tidal energy,cfd,gfd,roms,university of washington},
  pdftitle = {\name: Curriculum Vitae},
  pdfsubject = {Curriculum Vitae},
  pdfpagemode = UseNone
}

\geometry{
  body={6.75in, 9.0in},
  left=0.85in,
  top=1.0in
}

% Customize page headers
\pagestyle{myheadings}
\markright{\name}
\thispagestyle{empty}

% Custom section fonts
\usepackage{sectsty}
\sectionfont{\rmfamily\mdseries\Large}
\subsectionfont{\rmfamily\mdseries\itshape\large}

% Other possible font commands include:
% \ttfamily for teletype,
% \sffamily for sans serif,
% \bfseries for bold,
% \scshape for small caps,
% \normalsize, \large, \Large, \LARGE sizes.

% Don't indent paragraphs.
\setlength\parindent{0em}

% Make lists without bullets and compact spacing
\renewenvironment{itemize}{
  \begin{list}{}{
    \setlength{\leftmargin}{1.5em}
    \setlength{\itemsep}{0.25em}
    \setlength{\parskip}{0pt}
    \setlength{\parsep}{0.25em}
  }
}{
  \end{list}
}

\begin{document}
\input{../../LatexFiles/macros}

% Place name at left
{\huge \name}

% Alternatively, print name centered and bold:
%\centerline{\huge \bf \name}

\vspace{0.25in}

\begin{minipage}[t]{0.5\textwidth}
  \href{http://www.tamu.edu/}{Texas A\&M University} \\
  \href{http://ocean.tamu.edu/}{Department of Oceanography} \\
  1204 Eller O\&M Building \\
  3146 TAMU \\
  College Station, TX 77843-3146 \\
\end{minipage}
\begin{minipage}[t]{0.5\textwidth}
  Eller O\&M Building, Room 607 \\
  (979) 845-0791 \\
  \href{mailto:kthyng@tamu.edu}{\tt kthyng@tamu.edu} \\
  \href{http://kristenthyng.com}{\tt http://kristenthyng.com} \\
\end{minipage}

\section*{Education}

\begin{itemize}

  \item Ph.D. Mechanical Engineering, University of Washington, June 2012.

    \begin{itemize}
	\item   ``Numerical Simulation of Admiralty Inlet, WA, with Tidal Hydrokinetic Turbine Siting Application''
    \item \textit{Committee:}
      \href{http://faculty.washington.edu/rileyj/}{James J. Riley} (chair),
      \href{http://www.me.washington.edu/research/faculty/aaliseda/}{Alberto Aliseda},
      \href{http://faculty.washington.edu/kawase/}{Mitsuhiro Kawase},
      \href{http://www.me.washington.edu/research/faculty/bpolagye/}{Brian Polagye}, and
      \href{http://www.atmos.washington.edu/~durrand/}{Dale Durran}.

    \end{itemize}

  \item M.Sc. Applied Mathematics, University of Washington, 2007.

  \item B.A. Physics, Whitman College, 2005.

    \begin{itemize}

    \item \textit{Minor:} Mathematics, \textit{Honors:} \href{https://www.whitman.edu/content/catalog/financial-aid}{Walter Brattain Scholarship}, \textit{Study Abroad:}
      \href{http://www.semesteratsea.org/}{Semester at Sea},
      Fall 2001.

    \end{itemize}
%    \begin{itemize}
%
%    \item \textit{Minor:} Mathematics.
%
%    \item \textit{Honors:} \href{https://www.whitman.edu/content/catalog/financial-aid}{Walter Brattain Scholarship}.
%
%    \item \textit{Study Abroad:}
%      \href{http://www.semesteratsea.org/}{Semester at Sea},
%      Fall 2001.
%
%    \end{itemize}

\end{itemize}

% \section*{Experience}
\section*{Research Experience}

\subsection*{Texas A\&M University, Department of Oceanography}

\begin{itemize}

\item Assistant Research Professor, 2016--Present.
\item Assistant Research Scientist, 2015--2016.
\item Postdoctoral Research Associate, 2012--2015.

\end{itemize}

\subsection*{University of Washington, Department of Mechanical Engineering}

\begin{itemize}

\item Research Assistant, 2007--2012.

%\item Teaching Assistant,
%  \href{http://www.washington.edu/students/crscat/meche.html#me323}{Thermodynamics, ME 323},
%  \href{http://www.me.washington.edu/research/faculty/malte/}{Phil Malte},
%  Fall 2007.

\end{itemize}

% \subsection*{Prometheus Energy Co.}

% \begin{itemize}

% \item Jr. Scientist,
%   \href{http://www.prometheus-energy.com/aboutus/team.php}{John A. Barclay},
%   Summer 2006.

% \end{itemize}

% \subsection*{Whitman College,
%   Physics Department}
%
% \begin{itemize}
%
% \item Lab assistant: Summer 2003, Fall 2003--Spring 2004, Summer 2004
%
% % \item Fairbank Physics Research Assistant,
% %   \href{http://people.whitman.edu/~hoffman/}{Kurt Hoffman},
% %   Summer 2004.
%
% % \item First-Year Physics Lab Assistant,  Fall 2003--Spring 2004.
%
% % \item First-Year Physics Lab Reorganization,
% %  \href{http://people.whitman.edu/~beckmk/}{Mark Beck},
% %  Summer 2003.
%
% \end{itemize}

\section*{Research}

\subsection*{Publications}

\subsubsection*{Peer-reviewed}

\begin{itemize}

\item Dukhovskoy, D.S., S.L. Morey, E.P. Chassignet, X. Chen, V.J. Coles, L. Cui, C.K. Harris, R.D. Hetland, T. Hsu, A.J. Manning, M.R. Stukel, \kmt, and J. Wang (2021). Development of the CSOMIO coupled ocean-oil-sediment-biology model. \textit{Frontiers in Marine Science, Marine Pollution}. (submitted)

% \item Fiorendino, J.M., C. Gaonkar, D.W. Henrichs, \kmt, L. Campbell (2020). Coastal plankton community composition and diversity after a massive flooding event: the response to Hurricane Harvey. \textit{Limnology and Oceanography}. (submitted)

\item Steffen, J.M., B.A. Summers, T.M. Conway, \kmt, R.M. Sherrell, C.R. German, J.N. Fitzsimmons (2021). Short residence times for hydrothermally-sourced dissolved iron in the deep ocean. \textit{Nat. Geosci}. (under revision)

\item \tkm, D. Kobashi, V. Ruiz-Xomchuk, L. Qu, X. Chen, and R.D. Hetland (2021). Performance of offline passive tracer advection in the Regional Ocean Modeling System (ROMS; v3.6, revision 904). \textit{Geosci. Model Dev.}., 14, 391--407, \href{https://doi.org/10.5194/gmd-14-391-2021}{doi: 10.5194/gmd-14-391-2021}.

\item \kmt, The Importance of Colormaps, \textit{Computing in Science & Engineering}, vol. 22, no. 5, pp. 96-102, 1 Sept.-Oct. 2020, \href{https://doi.org/10.1109/MCSE.2020.3006946}{doi: 10.1109/MCSE.2020.3006946}. (invited)

\item \tkm, Hetland, R.D., Socolofsky, S.A., Fernando, N., Turner, E. L., Schoenbaechler, C. (2020). Hurricane Harvey Caused Unprecedented Freshwater Inflow to Galveston Bay. \textit{Estuaries and Coasts}, 1--17. \href{https://doi.org/10.1007/s12237-020-00800-6}{doi:10.1007/s12237-020-00800-6}.

\item Kealoha, A.K., Shamberger, K.E.F., DiMarco, S.F., \tkm, Hetland, R.D., Manzello, D.P., Slowey, N., Enochs, I. (2020). Surface Water CO2 Variability in the Gulf of Mexico (1996-2017). \textit{Sci. Rep} 10, 12279 (2020). \href{https://doi.org/10.1038/s41598-020-68924-0}{doi:10.1038/s41598-020-68924-0}.

\item Harrison, T., \tkm, & Polagye, B. (2020). Comparative evaluation of volumetric current measurements in a tidally-dominated, coastal setting: a virtual field experiment. \textit{Journal of Atmospheric and Oceanic Technology}.  \href{http://doi.org/10.1175/jtech-d-19-0131.1}{doi:10.1175/jtech-d-19-0131.1}.

\item \tkm~(2019). Deepwater Horizon Oil could have naturally reached Texas beaches. \textit{Marine Pollution Bulletin}, 149, 110527.

\item Greene, C.A.,  K. Thirumalai,  K. A. Kearney,  J. Miguel Delgado,  W. Schwanghart,  N. S. Wolfenbarger,  \kmt,  D. E. Gwyther,  A. S. Gardner,  D. D. Blankenship (2019). The Climate Data Toolbox for MATLAB.  \textit{Geochemistry, Geophysics, Geosystems}. \href{https://agupubs.onlinelibrary.wiley.com/doi/abs/10.1029/2019GC008392}{doi:10.1029/2019GC008392}.

\item Feng, D., B. R. Hodges, S. S. Socolofsky, \& \kmt~(2019). Tidal eddies at a narrow channel inlet in operational oil spill models.  \textit{Marine Pollution Bulletin}, 140, 374--387. \href{https://www.sciencedirect.com/science/article/pii/S0025326X19300712?via%3Dihub}{doi:10.1016/j.marpolbul.2019.01.051}.

\item \tkm~and R. D. Hetland (2018). Seasonal and interannual cross-shelf transport in Texas and Louisiana. \textit{Continental Shelf Research}, 160, 23-35. \href{https://www.sciencedirect.com/science/article/pii/S0278434317302509}{doi:10.1016/j.csr.2018.03.006}.

\item Khade, V., J. Kurian, P. Chang, I. Szunyogh, \kt, \& R. Montuoro. (2017). Oceanic Ensemble Forecasting in the Gulf of Mexico: An application to the case of the Deep Water Horizon oil spill. \textit{Ocean Modelling}, 113, 171--184.  \href{http://www.sciencedirect.com/science/article/pii/S1463500317300525}{doi:10.1016/j.ocemod.2017.04.004}.

\item \tkm~and R. D. Hetland (2017). Texas and Louisiana coastal vulnerability and shelf connectivity. \textit{Marine Pollution Bulletin}, Volume 116, Issues 1--2, Pages 226--233. \href{http://www.sciencedirect.com/science/article/pii/S0025326X16310773}{doi:10.1016/j.marpolbul.2016.12.074}.

\item Fitzsimmons, J. N., T. M. Conway, J.-M. Lee, R. Kayser, \kmt, S. G. John, and E. A. Boyle (2016), Dissolved iron and iron isotopes in the Southeastern Pacific Ocean, \textit{Global Biogeochem. Cycles}, 30,\\ \href{http://onlinelibrary.wiley.com/doi/10.1002/2015GB005357/full}{doi:10.1002/2015GB005357}.

\item \tkm, C. A. Greene, R. D. Hetland, H. M. Zimmerle, and S. F. DiMarco (2016). True colors of oceanography: Guidelines for effective and accurate colormap selection. \textit{Oceanography} 29(3):9--13,\\ \href{http://tos.org/oceanography/assets/docs/29-3_thyng.pdf}{doi:10.5670/oceanog.2016.66}.

\item Bacosa, H. P., \kmt, S. Plunkett, D. L. Erdner, Z. Liu (2016). The tarballs on Texas beaches following the 2014 Texas City ``Y'' Spill: Modeling, chemical, and microbiological studies, \textit{Marine Pollution Bulletin}, Volume 109, Issue 1, 15 August 2016, Pages 236-244. \href{http://www.sciencedirect.com/science/article/pii/S0025326X16303964}{doi:10.1016/j.marpolbul.2016.05.076}.

\item Roc, T, S. W. Funke, \kmt~ (2015). Standard methodology for tidal array project optimisation: An idealized study of the Minas Passage. \textit{Proceedings European Wave and Tidal Energy Conference}. Nantes, France.

\item \tkm~ and R. D. Hetland (2014). ``TracPy: Wrapping the Fortran Lagrangian trajectory model TRACMASS'' \textit{Proceedings of the 13th Python in Science Conference (SciPy 2014)}. \href{http://conference.scipy.org/proceedings/scipy2014/pdfs/thyng.pdf}{\url{https://doi.org/10.25080/Majora-14bd3278-00d}}.

\item Roc, T., D. Greaves, \kmt, D. Conley (2014). Tidal turbine representation in an ocean circulation model: Towards realistic applications. \textit{Ocean Engineering}, 78, 95--111. \href{http://dx.doi.org/10.1016/j.oceaneng.2013.11.010}{doi:10.1016/j.oceaneng.2013.11.010}.

\item \tkm, R. D. Hetland, M. T. Ogle, X. Zhang, F. Chen, \& L. Campbell (2013). Origins of \textit{Karenia brevis} harmful algal blooms along the Texas coast. \textit{Limnology \& Oceanography: Fluids \& Environments}, 3, 269-278. \href{http://lofe.dukejournals.org/content/3/269.full}{doi: 10.1215/21573689-2417719}.

\item \tkm, J. J. Riley, \& J. Thomson (2013). Inference of turbulence parameters from a ROMS simulation using the $k$-$\varepsilon$ closure scheme. \textit{Ocean Modelling}, 72(C), 104--118. \href{http://www.sciencedirect.com/science/article/pii/S1463500313001613}{doi: 10.1016/j.ocemod.2013.08.008}.

\item \tkm~ \& T. Roc. (2013). Tidal current turbine power capture and impact in an idealised channel simulation. \textit{Proceedings European Wave and Tidal Energy Conference}. Aalborg, Denmark.

\item Roc, T., \kmt, \& D. Conley (2011). Applying a numerical decision-making tool for tidal current turbine (TCT) planning projects to the Puget Sound estuary - Early Results. \textit{Proceedings European Wave and Tidal Energy Conference}. Southampton, UK.

\item Kawase, M., \& \kmt~ (2010). Three-dimensional hydrodynamic modelling of inland marine waters of Washington State, United States, for tidal resource and environmental impact assessment. \textit{Renewable Power Generation, IET}, 4(6), 568--578. \href{http://digital-library.theiet.org/content/journals/10.1049/iet-rpg.2009.0195}{doi:10.1049/iet-rpg.2009.0195}.

\end{itemize}

\subsubsection*{Other and Products}

\begin{itemize}

\item Barba, L. A., J. Bazan, J. Brown, R. V. Guimera, M. Gymrek, A. Hanna, L. J. Heagy, K. D. Huff, D. S. Katz, C. R. Madan, K. M. Moerman, K. E. Niemeyer, J. L. Poulson, P. Prins, K. Ram, A. Rokem, A. M. Smith, G. K. Thiruvathukal, \kmt, L. Uieda, B. E. Wilson, Y. Yehudi (2019). Giving software its due through community-driven review and publication. \textit{OSF Preprints}. \href{https://doi.org/10.31219/osf.io/f4vx6}{doi:10.31219/osf.io/f4vx6}.

\item M. T. Sherman, R. Blaylock, K. Lucas, M. E. Capron, J. R. Stewart, S. F. DiMarco, \kt, R. Hetland, M. H. Kim, C. Sullivan, Z. Moscicki, I. Tsukrov, M. R. Swift, M. D. Chambers, S. C. James, M. Brooks, B. von Herzen, A. Jones, D. Piper (2018). SeaweedPaddock: Initial Modeling and Design for a Sargassum Ranch. \textit{Proceedings of Oceans 18 Conference}, Charleston, NC, October 22--25.

\item \tkm~and M. Marta-Almeida (2017). ``Texas Automated Buoy System,'' Renovated website for Texas General Land Office. \verb+http://pong.tamu.edu/tabswebsite/+.

\item \tkm~ (2015). cmocean: Beautiful colormaps for oceanography. \\ \verb+https://github.com/matplotlib/cmocean.+

\item \tkm, C. H. Barker, K. Jordahl, D. Cherian (2014). TracPy, Zenodo, doi:10.5281/zenodo.10433.

\item \tkm~ (2012). Numerical Simulation of Admiralty Inlet, WA, with Tidal Hydrokinetic Turbine Siting Application (Doctoral dissertation).

\item \tkm, \& Riley, J. J. (2010, September). Idealized headland simulation for tidal hydrokinetic turbine siting metrics. In OCEANS 2010 (pp. 1-6). IEEE.

\end{itemize}

\subsection*{Conference and Seminar Presentations}
* indicates student presenter
\begin{itemize}

\item \kmt, V. Ruiz Xomchuk, D. Kobashi, L. Qu, R. D. Hetland, D. Dukhovskoy, S. Morey, X. Chen, E. Chassignet, ``Offline Tracer Advection in a Realistic Regional Ocean Model,'' Ocean Sciences Meeting, San Diego, CA, February 17, 2020.

\item \kmt, R. Hetland, N. Fernando, E. Turner, C. Schoenbaechler, S. Socolofsky. ``Freshwater inflow to Galveston Bay due to Hurricane Harvey,'' Resilience Rising: Research and Practice on Harvey and Hazards of the Future, Texas A\&M University, TX, September 6, 2019.

\item A. Freddo$^*$, S. Socolofsky, B. R. Hodges, \kt, D. Feng. ``Advanced Oil Spill Transport across the Bay/Coastal Boundary,'' 2019 Gulf of Mexico Oil Spill \& Ecosystem Science Conference, New Orleans, LA, February 4--7, 2019.

\item S. L. Morey, E. Chassignet, D. Dukhovskoy, M. Stukel, C. Harris, R. Hetland, \kt, V. Coles, T. Hsu, A. Manning, O. U. Mason. ``Development of a Coupled Modeling System for Simulating Oil-Microbial-Sediment
Interactions in the Ocean,'' 2019 Gulf of Mexico Oil Spill \& Ecosystem Science Conference, New Orleans, LA, February 4--7, 2019.

\item \kmt, R. Hetland, S. Socolofsky, N. Fernando, E. Turner, C. Schoenbaechler. ``Shelf transport and Hurricane Harvey flooding into Galveston Bay,'' Texas A\&M University--Galveston, Galveston, TX, October 17, 2018. \inv

\item \kmt, R. Hetland, N. Fernando, E. Turner, C. Schoenbaechler, S. Socolofsky. ``Freshwater inflow to Galveston Bay due to Hurricane Harvey,'' Physics of Estuaries and Coastal Seas, Galveston, TX, October 15, 2018.

\item \kmt, R. Hetland, K. Whilden, N. Fernando, E. Turner, C. Schoenbaechler, S. Socolofsky. ``Freshwater inflow to and around Galveston Bay due to Hurricane Harvey,'' Hurricane Harvey Research Symposium, Port Aransas, TX, August 23, 2018.

\item \kmt~and M. Marta-Almeida, ``Website for interacting with oceanographic data and numerical model output,'' SciPy Conference 2018, Austin, TX, July 12, 2018.

\item \kmt, ``Along-coast connectivity in Texas and Louisiana,'' Ocean Sciences, Portland, OR, February 12--16, 2018.

\item \kmt, ``Effective and Accurate Colormap Selection,'' American Geophysical Union (AGU) Fall Meeting, San Francisco, CA, December 15, 2016. \inv

\item \kmt, ``Custom Colormaps for Your Field,'' PLOTCON, New York, New York, November 17, 2016. \inv

\item \kmt, ``Shelf-bay-coast connectivity in the NW Gulf of Mexico,'' Physics of Estuaries and Coastal Seas (PECS), The Hague, The Netherlands, October 10, 2016.

\item \kmt~ and R. D. Hetland, ``Submesoscale eddies increase particle transport and dispersion over the Texas-Louisiana shelf,'' Liege Colloquium, Liege, Belgium, May 27, 2016.

\item \kmt, S. W. Funke, T. Roc, ``Tidal Farm Array Optimization: Dynamics, Engineering, and Environment,'' Ocean Sciences, New Orleans, LA, February 26, 2016.

\item \kmt~ and R. D. Hetland, ``Transport on and across the Texas shelf,'' Department of Oceanography, Texas A\&M University, College Station, TX, October 19, 2015.

\item \kmt~ and R. D. Hetland, ``Transport on and across the Texas shelf,'' Department of Marine and Coastal Sciences, Rutgers University, New Brunswick, NJ, October 12, 2015. \inv

\item S. W. Funke, \kmt, T. Roc, ``Standard methodology for tidal array project optimisation: An idealized study of the Minas Passage,'' 11th European Wave and Tidal Energy Conference, Nantes, France, September 9, 2015.

\item \kmt~ and R. D. Hetland, ``Texas-Louisiana Shelf and Coast Connectivity,'' Lagrangian Analysis and Prediction of Coastal and Ocean Dynamics, Winter Harbor, ME, July 29, 2015.

\item \kmt, ``Perceptual colormaps in matplotlib for oceanography,'' SciPy Conference 2015, Austin, TX, July 10, 2015.

\item \kmt, Simon W. Funke, Thomas Roc, ``Python in tidal energy: three tools used in a collaboration on array optimization,'' SciPy Conference 2015, Austin, TX, July 10, 2015.

\item \kmt~ and R. D. Hetland, ``Texas and Louisiana coastline sensitivity and oil dispersion,'' 2015 Gulf of Mexico Oil Spill \& Ecosystem Science Conference, Houston, TX, February 19, 2015.

\item \kmt~ and R. D. Hetland, ``Cross-shelf transport and dispersion due to baroclinic instabilities,'' Physics of Estuaries and Coastal Seas (PECS), Porto de Galinhas, Brazil, October 21, 2014.

\item \kmt, ``Perceptions of matplotlib colormaps,'' SciPy Conference 2014, Austin, TX, July 10, 2014.

\item \kmt~ and R. D. Hetland, ``TracPy: Wrapping the FORTRAN Lagrangian trajectory model TRACMASS,'' SciPy Conference 2014, Austin, TX, July 10, 2014.

\item \kmt~ and R. D. Hetland, ``What leads to transport in the northwestern Gulf of Mexico?,'' NASA MPOWIR speaker series, Goddard Space Flight Center, Greenbelt, MD, May 28, 2014. \textit{(Selected seminar speaker)}

\item \kmt~ and J. J. Riley, ``Tidal Hydrokinetic Energy and Site Characterization,'' Department of Geology, Texas A\&M University, November 22, 2013.

\item \kmt~ and R. D. Hetland, ``Effect of interannual and seasonal variability on oil fate along the Texas coastline,'' Estuarine and Coastal Modeling Conference, San Diego, CA, November 4, 2013.

\item \kmt, R. D. Hetland, and L. Campbell, ``Physical mechanism for \textit{Karenia brevis} bloom initiation in Texas,'' 7th Symposium on Harmful Algae, Sarasota, FL, October 29, 2013.

\item \kmt~ and R. D. Hetland, ``Particle tracking on a structured numerical grid and applications in the Gulf of Mexico,'' 12th International workshop on Multi-scale (Un)-structured mesh numerical Modeling for coastal, shelf, and global ocean dynamics (IMUM), University of Texas, Austin, September 17, 2013.

\item Panelist on best practices in data visualization, SciPy Conference, Austin, TX, June 27, 2013. \inv

\item \kmt, J. J. Riley, and M. Kawase, ``Vorticity Dynamics in Admiralty Inlet, Puget Sound,'' Gordon Research Seminar: Coastal Ocean Circulation, University of New England, June 9, 2013. \inv

\item \kmt, R. D. Hetland, X. Zhang, L, Campbell, ``Origins of Harmful Algal Blooms Along the Texas Coast,'' ASLO Aquatic Sciences Meeting, New Orleans, LA, February 21, 2013.

\item \kmt~ and J. J. Riley, ``Turbulence Modeling in a Numerical Model for Tidal Hydrokinetic Energy Siting,'' Texas A\&M University, October 13, 2011. \textit{(Invited)}

\item T. Roc, \kmt, D. Conley, ``Applying a numerical decision-making tool for tidal current turbine (TCT) planning projects to the Puget Sound estuary - Early Results,'' 8th European Wave and Tidal Energy Conference, Southampton, 2011.

% \item K. M. Thyng and J. J. Riley, \href{http://froude.me.washington.edu/presentations/nnmrec050511.pdf}{``Site Modeling for Tidal Turbines''}, 2nd Annual OSU-UW Northwest National Marine Renewable Energy Center Conference, University of Washington, May 5, 2011.

% \item K. M. Thyng and J. J. Riley, ``Understanding New Admiralty Inlet Simulation'', MoSSea Users Group, School of Oceanography, University of Washington, May 4, 2011.

% \item K. M. Thyng and J. J. Riley, ``Modeling for Tidal Energy Analysis'', MoSSea Users Group, School of Oceanography, University of Washington, January 19, 2011.

% \item K. M. Thyng and J. J. Riley, ``Tidal Energy and Turbine Siting Metrics'', Mechanical Engineering Student Seminar, University of Washington, October 11, 2010.

\item \kmt~ and J. J. Riley, ``Idealized Headland Simulation for Tidal Hydrokinetic Turbine Siting Metrics'', OCEANS 2010 MTS/IEEE Seattle, September 21, 2010.

\item \kmt~ and J. J. Riley, ``Working Toward Numerical Simulations of Admiralty Inlet for Tidal Hydrokinetic Energy,'' \href{http://inore.org/}{4th Annual INORE Symposium, Dartmouth, UK, May 12, 2010}.

\item \kmt~ and J. J. Riley, ``Tidal Energy in the Puget Sound'', SIAM UW, April 21, 2009.

% \item K. M. Thyng and J. J. Riley, ``Tidal Energy in the Puget Sound'', SIAM UW, May 29, 2008.

\end{itemize}

\subsection*{Poster Presentations}
* indicates student presenter
\begin{itemize}

\item \kmt, R. Hetland, S. Socolofsky, N. Fernando, E. Turner, C. Schoenbaechler, ``Hurricane Harvey Caused Unprecedented Freshwater Inflow to Galveston Bay,'' Gordon Research Conference: Coastal Ocean Dynamics, Southern New Hampshire University, June 16--20, 2019.

\item A. M. Smith, L. A. Barba, D. S. Katz, K. E. Niemeyer, T. Allard, J. Bazan, J. Brown, J. Clark, R. Valls Guimera, M. Gymrek, L. Heagy, K. Huff, G. K. Thiruvathukal, C. R. Madan, K. M. Moerman, L. Pantano, V. Pons, J. Poulson, P. Prins, K. Ram, E. Ramirez, A. Rokem, \kt, and Y. Yehudi. ``Minisymposterium: The Journal of Open Source Software,'' SIAM Computational Science and Engineering (CSE) 2019, Spokane, Washington, February 25--March 1, 2019.

\item X. Diao$^*$ and \kt. ``Defining the Mississippi River Plume with numerical drifters,'' 2019 Gulf of Mexico Oil Spill \& Ecosystem Science Conference, New Orleans, LA, February 4--7, 2019.

\item X. Diao$^*$ and \kmt, ``Defining the Mississippi river plume with numerical drifters,'' Physics of Estuaries and Coastal Seas, Galveston, TX, October 14--19, 2018.

\item C. M. Morabito-Gonzalez$^*$, \kmt, S.W. Funke, ``Analysis of Hydrodynamic Impact Induced by Tidal Turbine Arrays,'' Ocean Sciences, Portland, OR, February 6--8, 2018.

\item \kmt, S. Socolofsky, K. Whilden, ``Measuring Freshwater Exports from Galveston Bay after Hurricane Harvey,'' Ocean Sciences, Portland, OR, February 6--8, 2018.

\item \kmt, ``Using Satellite Images to Characterize the Galveston Bay Tidal Plume,'' Gulf of Mexico Oil Spill \& Ecosystem Science Conference (GoMOSES), New Orleans, LA, February 12--16, 2018.

\item L. Campbell, D. Henrichs, \kmt, ``Expanding the Network of Imaging FlowCytobots for Early Warning of HABs,'' 9th US Symposium on Harmful Algae, Baltimore, MD, November 11--17, 2017.

\item \kmt, D. Feng, B. Hodges, ``When does material exit Galveston Bay?,'' Gordon Research Conference: Coastal Ocean Dynamics, University of New England, June 11--16, 2017.

\item \kmt, ``Perceptual colormaps in matplotlib with an application in oceanography,'' SciPy Conference 2015, Austin, TX, July 8, 2015.

\item \kmt~ and R. D. Hetland, ``Texas-Louisiana Shelf and Coast Connectivity,'' Gordon Research Conference: Coastal Ocean Modeling, University of New England, June 7--12, 2015.

\item T. Roc, \kmt, and S. W. Funke, ``Benchmarking Tidal Array Optimization: a Balance between Impacts \& Economics of the Bay of Fundy - Early Results,'' 5th International Conference on Ocean Energy, November 4--6, 2014, Halifax, Canada.

\item \kmt~ and R. D. Hetland, ``Cross-shelf transport and dispersion due to baroclinic instabilities,'' European Geosciences Union General Assembly 2014, April 27--May 2, Vienna, Austria.

\item \kmt~ and R. D. Hetland, ``Texas-Louisiana Cross-shelf Transport due to Submesoscale Eddies,'' Ocean Sciences Meeting, Honolulu, Hawaii, February 23--28, 2014.

\item J. Kuehl, \kmt, and P. Chapman, ``GISR Drift Card Program: Surface Transport Observation,'' Gulf of Mexico Oil Spill and Ecosystem Science Conference, Mobile, Alabama, January 26--29, 2014.

\item \kmt~ and R. D. Hetland, ``Texas-Louisiana Shelf Connectivity and Time Variability using Particle Tracking,'' Gulf of Mexico Oil Spill and Ecosystem Science Conference, Mobile, Alabama, January 26--29, 2014.

\item \kmt~ and T. Roc, ``Tidal current turbine power capture and impact in an idealised channel simulation,'' 10th European Wave and Tidal Energy Conference, Aalborg, Denmark, September 2--5, 2013.

\item \kmt, J. J. Riley, and M. Kawase, ``Vorticity Dynamics in Admiralty Inlet, Puget Sound,'' Gordon Research Conference: Coastal Ocean Circulation, University of New England, June 9--14, 2013.

\item \kmt, J. J. Riley, and J. Thomson, \href{http://pong.tamu.edu/~kthyng/posters/pecs.pdf}{ROMS Turbulence Parameter Comparisons with Field Data}, The Physics of Estuaries and Coastal Seas (PECS) Symposium, 12--16 August 2012, New York City.

\item \kmt~ and J. J. Riley, Nested ROMS Model of a Complex Estuarine Channel, Puget Sound, WA. Gordon Research Conference: Coastal Ocean Modeling,
  Mt. Holyoke College, South Hadley,  MA,
  June 26--July 1, 2011.

\item \kmt~ and J. J. Riley, \href{http://froude.me.washington.edu/presentations/GPSS051311.pdf}{Site Modeling for Tidal Turbines}. Graduate and Professional Student Senate Science and Policy Summit, University of Washington, May 13, 2011.

\item \kmt~ and J. J. Riley, \href{http://froude.me.washington.edu/presentations/InorePoster2010/inore.pdf}{Numerical Modeling for Tidal
Hydrokinetic Turbine Siting}. \href{http://www.inore.org}{4th Annual INORE Symposium, Dartmouth, UK, May 9, 2010}.

\item \kmt~ and J. J. Riley, \href{http://froude.me.washington.edu/presentations/inore_poster2009.pdf}{Estuary Modeling for Tidal Energy in Puget Sound, WA}. \href{http://www.inore.org}{3rd Annual INORE Symposium, Gent, Belgium, May 26, 2009}.

\end{itemize}

\subsection*{Grants}
\begin{itemize}
    \item Developing UAV and Satellite Tools for Ocean Currents and Oil Transport, Texas General Land Office, September 1, 2019 -- August 31, 2021, \$332,613, PI: S. A. Socolofsky, co-PIs: K. M. Thyng, B. R. Hodges, K. Chang.

    \item Ocean Energy from Macro Algae, ARPA-E, Department of Energy, June 29, 2018--June 28, 2019. PI: S.F. DiMarco, co-PI: K. M. Thyng \$25,000.

    \item Seaweed Paddock, ARPA-E, Department of Energy, June 29, 2018--June 28, 2019. PI: S.F. DiMarco, co-PIs: T. Knap and K. M. Thyng \$45,000; University of Southern Mississippi is lead institution.

\item Consortium for Simulation of Oil-Microbial Interactions in the Ocean (CSOMIO), Gulf of Mexico Research Institute, January 1, 2018--December 31, 2019. co-PIs: K. M. Thyng and R. D. Hetland at TAMU for \$165,128; Florida State University is lead institution.

  \item Measuring freshwater exports from Galveston Bay after Hurricane Harvey, NSF RAPID, October 15, 2017 -- September 30, 2019, \$134,964, PI: K. M. Thyng, co-PI: S. Socolofsky.

\item Improving Hydrodynamic Predictions of Surface Currents Near the Texas Coast Used for Rapid Oil Spill Response -- Phase 5, Texas General Land Office, September 1, 2017 -- August 31, 2019, \$438,591, PI: R. D. Hetland, co-PI: K. M. Thyng.
    \item Extending and Improving Texas Bay/Estuary Oil Spill Simulations, Texas        General Land Office, September 1, 2017 -- August 31, 2019, \$463,738, PI: B. R. Hodges, co-PIs: S. A. Socolofsky, K. M. Thyng.
  \item Design of a Modern Web Interface to TGLO TABS Model and Data Products -- Phase 2, Texas General Land Office, September 1, 2015 -- August 31, 2017, \$186,988, PI: K. M. Thyng, co-PI: R. D. Hetland.
  \item Improving Oil Spill Predictions Near Shore and Across The Bay/Coastal Interface, Texas General Land Office, September 1, 2015 -- August 31, 2017, \$406,910, PI: B. R. Hodges, co-PIs: S. A. Socolofsky, K. M. Thyng.
  \item Improving Hydrodynamic Predictions of Surface Currents Near the Texas Coast Used for Rapid Oil Spill Response -- Phase 4, Texas General Land Office, September 1, 2015 -- August 31, 2017, \$376,560, PI: R. D. Hetland, co-PI: K. M. Thyng.
\end{itemize}

% \subsection*{Other Funding Received}

% \begin{itemize}
%   % \item 7th U.S. Harmful Algae Symposium travel award (\$500) based on statement (2013).
%   \item \href{http://inore.org/news/icis_travel_grants/}{International Collaboration Incentive Scheme grant}, a collaborative research grant through the International Network for Offshore Renewable Energy (INORE) (2011).
%   % \item Travel and conference funding for Gordon Research Conference: Ocean Modeling (2011).
%   % \item Travel funding from the Graduate School Fund for Excellence and Innovation (GSFEI), University of Washington (2010).
%   % \item Symposium and travel funding, International Network of Offshore Renewable Energy researchers (2010)
%   % \item Symposium and travel funding, International Network of Offshore Renewable Energy researchers (2009)
%   % \item Energy Summer School funding, U.K. Energy Research Council (2008).
% \end{itemize}

\subsection*{Selected Other Conferences and Workshops Attended}

\begin{itemize}

\item Gulf of Mexico Reefs: Past, Present, and Future, Rice University,  Houston, TX, October 10--11, 2018.

\item MPOWIR Pattullo Conference for women in physical oceanography, Airlie Center, Warrenton, Virginia, October 6--9, 2013.

% \item The 12th International workshop on Multi-scale (Un)-structured mesh numerical Modeling for coastal, shelf, and global ocean dynamics (IMUM), University of Texas, Austin, September 16--19, 2013.

% \item Gordon Research Seminar and Gordon Research Conference: Coastal Ocean Circulation, University of New England, June 9--14, 2013.

% \item Association for the Sciences of Limnology and Oceanography (ASLO) 2013 Aquatic Sciences Meeting, February 17--22, 2013, New Orleans, Louisiana.

% \item Gulf Of Mexico: Oil Spill \& Ecosystem Science Conference, January 21--23, 2013, New Orleans, Louisiana.

\item Subsea Blowout Modeling Workshop, University of California, Berkeley, November 27--28, 2012.

% \item The Physics of Estuaries and Coastal Seas (PECS) Symposium, 12-16 August 2012, New York City.

% \item \href{http://www.grc.org/programs.aspx?year=2011&program=coastal}{Gordon Research Conference: Coastal Ocean Modeling},
  % Mt. Holyoke College, South Hadley,  MA,
  % June 26--July 1, 2011.

% \item \href{http://www.oceans10mtsieeeseattle.org/}{OCEANS 2010 MTS/IEEE},
%   Seattle,
%   September 20--23, 2010.

% \item \href{http://www.inore.org}{4th Annual International Network of Offshore Renewable Energy researchers (INORE) Symposium},
%   Dartmouth, UK,
%   May 9--14, 2010.

% \item \href{http://www.agu.org/meetings/os10/}{2010 Ocean Sciences Meeting},
  % Oregon Convention Center,
  % February 22--26, 2010.

% \item \href{http://www.oce.uri.edu/ecm11/index.html}{Eleventh International Conference on Estuarine and Coastal Modeling},
  % Seattle, WA,
  % November 4--6, 2009.

% \item \href{http://www.inore.org}{3rd Annual International Network of Offshore Renewable Energy researchers (INORE) Symposium},
%   Gent, Belgium,
%   May 24--28, 2009.

\item \href{http://www.ukerc.ac.uk/support/tiki-index.php?page=0608UKERCSummerSchool}{U.K. Energy Research Council (UKERC) Energy Summer School 2008},
  Roehampton, London, UK,
  June 23--27, 2008.

\end{itemize}

\section*{Teaching}

\subsection*{Texas A\&M University, Department of Oceanography}

\begin{itemize}
\item Python, Texas A\&M High Performance Research Computing, February 7, 2020 (80+ students local/online).
\item Introduction to Oceanography (OCNG 251), Spring 2020, undergraduate class (39 students).
\item Python for Geoscientists (OCNG 469/669), Fall 2019, undergrad/graduate class, with remote (27 students).
\item Python for Geoscientists (OCNG 469/669), Spring 2018, undergrad/graduate class (21 students).
\item Python for Geoscientists (OCNG 469/669), Fall 2017, undergrad/graduate class, with remote (26 students).
\item Python for Geoscientists (OCNG 469/669), Fall 2016, undergrad/graduate class, with remote (10 students).
\item Python for Geoscientists (OCNG 489/689), Spring 2016, undergrad/graduate class (18 students).
\item Introduction to Oceanography (OCNG 251), Fall 2015, undergraduate class (68 students).
\item Introduction to Oceanography (OCNG 251), Spring 2015, undergraduate class (46 students).
% \item Guest Lecturer: Introduction to Oceanography (OCNG 251), Spring 2014, undergraduate class.
% \item Guest Lecturer: Python for Geoscientists (OCNG 689), Fall 2013, graduate class.
% \item Guest Lecturer: Introduction to Physical Oceanography (OCNG 410), Spring 2013, undergraduate class.

\end{itemize}

\subsection*{University of Washington, Department of Mechanical Engineering}

\begin{itemize}

\item Teaching Assistant,
  \href{http://www.washington.edu/students/crscat/meche.html#me323}{Thermodynamics, ME 323},
  \href{http://www.me.washington.edu/research/faculty/malte/}{Dr. Philip Malte},
  Fall 2007.

\end{itemize}



\section*{Mentoring}

\subsection*{Member}

\begin{itemize}
\item Janelle Steffen, PhD, Oceanography, 2019--
\item Tianxiang Gao, PhD, Oceanography, 2019--
\item Tacey Hicks, PhD, Oceanography, 2018--
\item Meghan Daniels, MSc, Civil Engineering, 2018--2019
\item James Fiorendino, PhD, Oceanography, 2017--
\item Xiliang Diao, PhD, Oceanography, 2017--2019
\end{itemize}

\subsection*{Undergraduate}

\begin{itemize}
\item Research assistant: Kate Von Ness, Summer 2019. Visualized ship CTD and other data from post-Hurricane Harvey cruises.
\item Research assistant: Shelley Culver, Summer 2018. Visualized ship flow through data from post-Hurricane Harvey cruises.
\item REU Student: Molly Kerwick, Summer 2018. Modeled Lagrangian drifters in the northwest Gulf of Mexico to and from the Flower Garden Banks to examine transport pathways.
\item REU Student: Cassidy Gonzalez-Morabito, Summer 2017. Modeled impact to vertical vorticity from tidal turbines in a headland tidal channel.
\end{itemize}



\section*{Field Work}

\subsection*{Cruises}
\begin{itemize}
    \item Galveston Bay: R/V Trident, NSF RAPID, October 8, 2017 (included 1 graduate student)
    \item Texas continental shelf: R/V Point Sur, NSF RAPID, September 27--29, 2017 (included 5 graduate and 3 undergraduate students)
    \item Texas continental shelf: R/V Pelican, NSF RAPID, March 25--27, 2018 (included 4 graduate and 5 undergraduate students)
\end{itemize}

\subsection*{Additional Field Opportunities for Students}
\begin{itemize}
    \item Galveston Bay: R/V Trident, NSF RAPID, June 21, 2018 (2 graduate and 4 undergraduate students)
    \item Galveston Bay: R/V Trident, NSF RAPID, April 29--30, 2018 (1 graduate and 2 undergraduate students)
    \item Galveston Bay: R/V Trident, NSF RAPID, November 12--13, 2017 (2 graduate students)
    \item Texas continental shelf: R/V Point Sur, NSF RAPID, November 6--8, 2017 (included 5 graduate and 4 undergraduate students)
\end{itemize}


\section*{Honors \& Awards}

\begin{itemize}

\item Aggies Celebrate Teaching! Recognizing Transformational Teaching and Learning Teaching Award. Center for Teaching Excellence, Texas A\&M University, 2019. 6 recipients out of 100 nominations across the university and multiple campuses.

\item SciPy John Hunter Excellence in Plotting Competition 2015: Honorable mention.

\item SciPy John Hunter Excellence in Plotting Competition 2014: 3rd place.

\item NASA MPOWIR (Mentoring Physical Oceanography Women to Increase Retention) selected speaker, 2014.

\item SciPy John Hunter Excellence in Plotting Competition 2013: 2nd place.

\item Outstanding Female Award, Mechanical Engineering. Society of Women Engineers, University of Washington. January 25, 2012.

\item \href{http://www.inore.org}{Best Symposium Poster}, INORE Symposium, 2010.

% \item \href{https://www.whitman.edu/content/catalog/financial-aid}{Walter Brattain Scholarship}, 2001-2005.

%\item Holy Mother Marie Rose Scholarship, 1997-2001.
%
%\item Valedictorian, 2001

\end{itemize}

\section*{Service}
\begin{itemize}
  \item Co-chair of Earth, Ocean and Geoscience mini-symposium and program committee member, SciPy conference, 2020.
  \item Associate Editor-in-Chief of Journal of Open Source Software, 2019--present.
  \item Discussion leader, Coastal Ocean Dynamics Gordon Research Conference, 2019.
  \item Topic editor of Journal of Open Source Software, 2018--2019.
  \item Co-chair of Earth, Ocean and Geoscience mini-symposium and program committee member, SciPy conference, 2018.
  \item Panel: ``State of Diversity \& Inclusion in the SciPy Community,'' SciPy conference, 2017.
  \item Chair of Earth, Ocean and Geoscience mini-symposium, SciPy conference, 2017.
  \item NSF Panel Reviewer, Division of Graduate Education, 2017.
  \item Member of Program Committee, JupyterCon 2017.
  \item Pen pal with young female student in mathematics (2016).
  \item Associate Chair for Gordon Research Seminar on Coastal Ocean Modeling (2015).
  \item Co-chair for tutorials and on program committee, SciPy Conference 2014 and 2015.
  \item Organizer and fund raiser for diversity event at SciPy conference, 2013 and 2014.
  % \item Diversity Committee: co-organizer and fund raiser for event, SciPy Conference 2014.
  % % \item Co-chair for tutorials, SciPy Conference 2014.
  % \item Organizer and fund raiser for Women in Scientific Computing event, June 27, 2013.
	\item \href{http://www.amath.washington.edu/~siamuw/math-fair.html}{Math and Science Fair}, Lockwood Elementary School, December 15, 2010 and March 15, 2011; Emerson Elementary School, June 8, 2010.
	% \item \href{http://www.amath.washington.edu/~siamuw/math-fair.html}{Math and Science Fair}, Lockwood Elementary School, December 15, 2010.
	% \item Math and Science Fair, Emerson Elementary School, June 8, 2010.
	\item An Introduction to Tidal Energy Research in the Puget Sound. UW Robinson School Summer Challenge, July 19, 2010.
	% \item Volunteer at \href{http://www.globalmarinerenewable.com/}{3rd Annual Global Marine Renewable Energy Conference},  Bell Harbor International Convention Center,  April 14--15, 2010.
  \item Referee: PLOSone; Nature Energy; PNAS; Geophysical Research Letters; Journal of Geophysical Research -- Oceans; Ocean Modelling; GSA Today; Maryland Sea Grant; Estuaries \& Coasts; EuroSciPy; Journal of Marine Science and Engineering; NSF: Division of Ocean Sciences; Ocean Dynamics; Arctic and Marine Oilspill Program (AMOP) Technical Seminar on Environmental Contamination and Response; Packt Publishing; International Conference on Ocean, Offshore and Arctic Engineering (OMAE); National Oceanic and Atmospheric Administration; European Wave and Tidal Energy Conference
\end{itemize}


\subsection*{Professional Activities}

\begin{itemize}

\item Member, \href{http://www.agu.org/}{American Geophysical Union} (2010--Present); \href{http://www.siam.org/}{Society for Industrial and Applied Mathematics}, (2008--2013); \href{http://www.aslo.org/}{Association for the Sciences of Limnology and Oceanography} (2012--2013); \href{http://www.ieee.org/}{Institute of Electrical and Electronics Engineers} (2010--2012).

%\item Referee for:
%
%  \begin{itemize}
%  \item \textit{Economic Inquiry} % 2008 (1)
%  \item \textit{Journal of Business and Economic Statistics} % 2009 (1)
%  \item \textit{Journal of Econometrics} % 2009 (1)
%  \item \textit{Scandinavian Journal of Statistics} % 2009 (1)
%  \end{itemize}

\end{itemize}

\section*{Skills}
\begin{itemize}
	\item Extensively used ROMS ocean modeling code, past use of SUNTANS ocean modeling code.
	\item Expert in Python and \LaTeX; skilled in FORTRAN and MatLab; proficient in C and PHP.
	\item Extensive experience with Linux system administration, using a cluster, and parallel computing.
	\item Experience with HTML and CSS.
\end{itemize}

% \section*{Other Interests}
% \begin{itemize}
% 	\item Hiking
% 	\item Mountaineering
% 	\item Road and trail running
% \end{itemize}

% \bigskip

% Footer
\begin{center}
  \begin{small}
    Last updated: \today
  \end{small}
\end{center}

\end{document}
