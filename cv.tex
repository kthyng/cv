% Jason R. Blevins - Curriculum Vitae
%
% Copyright (C) 2004-2010 Jason R. Blevins
% http://jblevins.org/projects/cv-template/
%
% You may use use this document as a template to create your own CV
% and you may redistribute the source code freely. No attribution is
% required in any resulting documents. I do ask that you please leave
% this notice and the above URL in the source code if you choose to
% redistribute this file.

\documentclass[10pt,letterpaper]{article}

\usepackage{hyperref}
\usepackage{geometry}
\usepackage[T1]{fontenc}

% Comment the following line to use the default Computer Modern font
% instead of the Palatino font provided by the mathpazo package.
% Remove the 'osf' bit if you don't like the old style figures.
\usepackage[sc,osf]{mathpazo}

% In practice, I use the following font packages instead of mathpazo.
% beramono provides a nice fixed-width font. xagaramon uses the
% (commercial) Adobe Garamond font.
%\usepackage[scaled=0.75]{beramono}
%\usepackage[osf]{xagaramon}

% Set your name here
\def\name{Kristen M. Thyng}

% The following metadata will show up in the PDF properties
\hypersetup{
  colorlinks = true,
  urlcolor = black,
  pdfauthor = {\name},
  pdfkeywords = {tidal energy,cfd,gfd,roms,university of washington},
  pdftitle = {\name: Curriculum Vitae},
  pdfsubject = {Curriculum Vitae},
  pdfpagemode = UseNone
}

\geometry{
  body={6.75in, 9.0in},
  left=0.85in,
  top=1.0in
}

% Customize page headers
\pagestyle{myheadings}
\markright{\name}
\thispagestyle{empty}

% Custom section fonts
\usepackage{sectsty}
\sectionfont{\rmfamily\mdseries\Large}
\subsectionfont{\rmfamily\mdseries\itshape\large}

% Other possible font commands include:
% \ttfamily for teletype,
% \sffamily for sans serif,
% \bfseries for bold,
% \scshape for small caps,
% \normalsize, \large, \Large, \LARGE sizes.

% Don't indent paragraphs.
\setlength\parindent{0em}

% Make lists without bullets and compact spacing
\renewenvironment{itemize}{
  \begin{list}{}{
    \setlength{\leftmargin}{1.5em}
    \setlength{\itemsep}{0.25em}
    \setlength{\parskip}{0pt}
    \setlength{\parsep}{0.25em}
  }
}{
  \end{list}
}

\begin{document}
\input{../../LatexFiles/macros}

% Place name at left
{\huge \name}

% Alternatively, print name centered and bold:
%\centerline{\huge \bf \name}

\vspace{0.25in}

\begin{minipage}[t]{0.5\textwidth}
  \href{http://www.tamu.edu/}{Texas A\&M University} \\
  \href{http://ocean.tamu.edu/}{Department of Oceanography} \\
  3146 TAMU \\
  College Station, TX 77843-3146 \\
\end{minipage}
\begin{minipage}[t]{0.5\textwidth}
  (979) 845-1231 \\
  \href{mailto:kthyng@tamu.edu}{\tt kthyng@tamu.edu} \\
  \href{http://kristenthyng.com}{\tt http://kristenthyng.com} \\
\end{minipage}

\section*{Education}

\begin{itemize}

  \item Ph.D. Mechanical Engineering, University of Washington, June 2012.

    \begin{itemize}
	\item   ``Numerical Simulation of Admiralty Inlet, WA, with Tidal Hydrokinetic Turbine Siting Application''
    \item \textit{Committee:}
      \href{http://faculty.washington.edu/rileyj/}{James J. Riley} (chair),
      \href{http://www.me.washington.edu/research/faculty/aaliseda/}{Alberto Aliseda},
      \href{http://faculty.washington.edu/kawase/}{Mitsuhiro Kawase},
      \href{http://www.me.washington.edu/research/faculty/bpolagye/}{Brian Polagye}, and
      \href{http://www.atmos.washington.edu/~durrand/}{Dale Durran}.

    \end{itemize}

  \item M.Sc. Applied Mathematics, University of Washington, 2007.

  \item B.A. Physics, Whitman College, 2005.

    \begin{itemize}

    \item \textit{Minor:} Mathematics, \textit{Honors:} \href{https://www.whitman.edu/content/catalog/financial-aid}{Walter Brattain Scholarship}, \textit{Study Abroad:}
      \href{http://www.semesteratsea.org/}{Semester at Sea},
      Fall 2001.

    \end{itemize}
%    \begin{itemize}
%
%    \item \textit{Minor:} Mathematics.
%    
%    \item \textit{Honors:} \href{https://www.whitman.edu/content/catalog/financial-aid}{Walter Brattain Scholarship}.
%
%    \item \textit{Study Abroad:}
%      \href{http://www.semesteratsea.org/}{Semester at Sea},
%      Fall 2001.
%
%    \end{itemize}

\end{itemize}

\section*{Employment}
%\section*{Research Experience}

\subsection*{Texas A\&M University, Department of Oceanography}

\begin{itemize}

\item Assistant Research Scientist, 2015--Present.
\item Postdoctoral Research Associate, \href{http://pong.tamu.edu/~rob/}{Robert D. Hetland}, 2012--2015.

\end{itemize}

\subsection*{University of Washington, Department of Mechanical Engineering}

\begin{itemize}

\item Research Assistant,
  \href{http://faculty.washington.edu/rileyj/}{James J. Riley},
 2007--2012.

%\item Teaching Assistant,
%  \href{http://www.washington.edu/students/crscat/meche.html#me323}{Thermodynamics, ME 323},
%  \href{http://www.me.washington.edu/research/faculty/malte/}{Phil Malte},
%  Fall 2007.

\end{itemize}

\subsection*{Prometheus Energy Co.}

\begin{itemize}

\item Jr. Scientist,
  \href{http://www.prometheus-energy.com/aboutus/team.php}{John A. Barclay},
  Summer 2006.

\end{itemize}

\subsection*{Whitman College,
  Physics Department}

\begin{itemize}

\item Fairbank Physics Research Assistant,
  \href{http://people.whitman.edu/~hoffman/}{Kurt Hoffman},
  Summer 2004.

\item First-Year Physics Lab Assistant,  Fall 2003--Spring 2004.

\item First-Year Physics Lab Reorganization, 
 \href{http://people.whitman.edu/~beckmk/}{Mark Beck},
 Summer 2003.

\end{itemize}

\section*{Research}

\subsection*{Publications}

\subsubsection*{Peer-reviewed}

\begin{itemize}

% \item Bacosa, H. P., Thyng, K. M., Plunkett, S., Erdner, D. L., and Liu, Z. (submitted). Chemical Fingerprinting and Characterization, Bacterial Community Analysis, and Transport Modeling: An Integrated Approach to Describe the Fate of Tarballs Following the 2014 Texas City ``Y'' Oil Spill. \textit{Environmental Science \& Technology Letters}.

\item Thyng, K. M., Hetland, R. D., Zimmerle, H. M., DiMarco, S. F. (submitted). Choosing good colormaps: accurate and effective data visualization. \textit{Oceanography}.

\item Thyng, K. M. and R. D. Hetland (submitted). Texas and Louisiana Coastal Vulnerability and Shelf Connectivity. \textit{Marine Pollution Bulletin}.

\item Thyng, K. M. and R. D. Hetland (submitted). Mechanisms controlling seasonal and interannual cross-shelf transport in Texas and Louisiana. \textit{Journal of Geophysical Research -- Oceans}.

\item Roc, T, Funke, S. W., Thyng, K. M. (2015). Standard methodology for tidal array project optimisation: An idealized study of the Minas Passage. \textit{Proceedings European Wave and Tidal Energy Conference}. Nantes, France.

\item Thyng, K. M. and R. D. Hetland, \href{http://conference.scipy.org/proceedings/scipy2014/pdfs/thyng.pdf}{``TracPy: Wrapping the Fortran Lagrangian trajectory model TRACMASS''} \textit{Proceedings of the 13th Python in Science Conference (SciPy 2014)}.

\item Roc, T., Greaves, D., Thyng, K. M., Conley, D. (2014). Tidal turbine representation in an ocean circulation model: Towards realistic applications. \textit{Ocean Engineering}, 78, 95--111. \href{http://dx.doi.org/10.1016/j.oceaneng.2013.11.010}{doi:10.1016/j.oceaneng.2013.11.010}.

\item Thyng, K. M., Hetland, R. D., Ogle, M. T., Zhang, X., Chen, F., \& Campbell, L. (2013). Origins of \textit{Karenia brevis} harmful algal blooms along the Texas coast. \textit{Limnology \& Oceanography: Fluids \& Environments}, 3, 269-278. \href{http://lofe.dukejournals.org/content/3/269.full}{doi: 10.1215/21573689-2417719}.

\item Thyng, K. M., Riley, J. J., \& Thomson, J. (2013). Inference of turbulence parameters from a ROMS simulation using the $k$-$\varepsilon$ closure scheme. \textit{Ocean Modelling}, 72(C), 104--118. \href{http://www.sciencedirect.com/science/article/pii/S1463500313001613}{doi: 10.1016/j.ocemod.2013.08.008}.

\item Thyng, K. M. \& Roc., T. (2013). Tidal current turbine power capture and impact in an idealised channel simulation. \textit{Proceedings European Wave and Tidal Energy Conference}. Aalborg, Denmark.

\item Roc, T., Thyng, K. M., \& Conley, D. (2011). Applying a numerical decision-making tool for tidal current turbine (TCT) planning projects to the Puget Sound estuary - Early Results. \textit{Proceedings European Wave and Tidal Energy Conference}. Southampton, UK.

\item Kawase, M., \& Thyng, K. M. (2010). Three-dimensional hydrodynamic modelling of inland marine waters of Washington State, United States, for tidal resource and environmental impact assessment. \textit{Renewable Power Generation, IET}, 4(6), 568--578. doi:10.1049/iet-rpg.2009.0195.

\end{itemize}

\subsubsection*{Other and Products}

\begin{itemize}

\item Thyng, K. M. (2015). cmocean: Beautiful colormaps for oceanography. \\ \verb+https://github.com/matplotlib/cmocean.+

\item Thyng, K. M., C. H. Barker, K. Jordahl, D. Cherian (2014). TracPy, Zenodo, doi:10.5281/zenodo.10433.

\item Thyng, K. M. (2012). Numerical Simulation of Admiralty Inlet, WA, with Tidal Hydrokinetic Turbine Siting Application (Doctoral dissertation).

\item Thyng, K. M., \& Riley, J. J. (2010, September). Idealized headland simulation for tidal hydrokinetic turbine siting metrics. In OCEANS 2010 (pp. 1-6). IEEE.

\end{itemize}

\subsection*{Conference and Seminar Presentations}

\begin{itemize}

\item K. M. Thyng and R. D. Hetland, ``Transport on and across the Texas shelf,'' Department of Oceanography, Texas A\&M University, College Station, TX, October 19, 2015. \textit{(invited)}

\item K. M. Thyng and R. D. Hetland, ``Transport on and across the Texas shelf,'' Department of Marine and Coastal Sciences, Rutgers University, New Brunswick, NJ, October 12, 2015. \textit{(invited)}

\item S. W. Funke, K. M. Thyng, T. Roc, ``Standard methodology for tidal array project optimisation: An idealized study of the Minas Passage,'' 11th European Wave and Tidal Energy Conference, Nantes, France, September 9, 2015.

\item K. M. Thyng and R. D. Hetland, ``Texas-Louisiana Shelf and Coast Connectivity,'' Lagrangian Analysis and Prediction of Coastal and Ocean Dynamics, Winter Harbor, ME, July 29, 2015.

\item K. M. Thyng, ``Perceptual colormaps in matplotlib for oceanography,'' SciPy Conference 2015, Austin, TX, July 10, 2015.

\item K. M. Thyng, Simon W. Funke, Thomas Roc, ``Python in tidal energy: three tools used in a collaboration on array optimization,'' SciPy Conference 2015, Austin, TX, July 10, 2015.

\item K. M. Thyng and R. D. Hetland, ``Texas and Louisiana coastline sensitivity and oil dispersion,'' 2015 Gulf of Mexico Oil Spill \& Ecosystem Science Conference, Houston, TX, February 19, 2015.

\item K. M. Thyng and R. D. Hetland, ``Cross-shelf transport and dispersion due to baroclinic instabilities,'' Physics of Estuaries and Coastal Seas (PECS), Porto de Galinhas, Brazil, October 21, 2014.

\item K. M. Thyng, ``Perceptions of matplotlib colormaps,'' SciPy Conference 2014, Austin, TX, July 10, 2014.

\item K. M. Thyng and R. D. Hetland, ``TracPy: Wrapping the FORTRAN Lagrangian trajectory model TRACMASS,'' SciPy Conference 2014, Austin, TX, July 10, 2014.

\item K. M. Thyng and R. D. Hetland, ``What leads to transport in the northwestern Gulf of Mexico?,'' NASA MPOWIR speaker series, Goddard Space Flight Center, Greenbelt, MD, May 28, 2014. \textit{(Selected seminar speaker)}

\item K. M. Thyng and J. J. Riley, ``Tidal Hydrokinetic Energy and Site Characterization,'' Department of Geology, Texas A\&M University, November 22, 2013. \textit{(Invited)}

\item K. M. Thyng and R. D. Hetland, ``Effect of interannual and seasonal variability on oil fate along the Texas coastline,'' Estuarine and Coastal Modeling Conference, San Diego, CA, November 4, 2013.

\item K. M. Thyng, R. D. Hetland, and L. Campbell, ``Physical mechanism for \textit{Karenia brevis} bloom initiation in Texas,'' 7th Symposium on Harmful Algae, Sarasota, FL, October 29, 2013.

\item K. M. Thyng and R. D. Hetland, ``Particle tracking on a structured numerical grid and applications in the Gulf of Mexico,'' 12th International workshop on Multi-scale (Un)-structured mesh numerical Modeling for coastal, shelf, and global ocean dynamics (IMUM), University of Texas, Austin, September 17, 2013.

\item Panelist on best practices in data visualization, SciPy Conference, Austin, TX, June 27, 2013. \textit{(Invited)}

\item K. M. Thyng, J. J. Riley, and M. Kawase, ``Vorticity Dynamics in Admiralty Inlet, Puget Sound,'' Gordon Research Seminar: Coastal Ocean Circulation, University of New England, June 9, 2013. \textit{(Invited)}

\item K. M. Thyng, R. D. Hetland, X. Zhang, L, Campbell, ``Origins of Harmful Algal Blooms Along the Texas Coast,'' ASLO Aquatic Sciences Meeting, New Orleans, LA, February 21, 2013.

\item K. M. Thyng and J. J. Riley, ``Turbulence Modeling in a Numerical Model for Tidal Hydrokinetic Energy Siting,'' Texas A\&M University, October 13, 2011. \textit{(Invited)}

\item T. Roc, K. M. Thyng, D. Conley, ``Applying a numerical decision-making tool for tidal current turbine (TCT) planning projects to the Puget Sound estuary - Early Results,'' 8th European Wave and Tidal Energy Conference, Southampton, 2011.

% \item K. M. Thyng and J. J. Riley, \href{http://froude.me.washington.edu/presentations/nnmrec050511.pdf}{``Site Modeling for Tidal Turbines''}, 2nd Annual OSU-UW Northwest National Marine Renewable Energy Center Conference, University of Washington, May 5, 2011.

% \item K. M. Thyng and J. J. Riley, ``Understanding New Admiralty Inlet Simulation'', MoSSea Users Group, School of Oceanography, University of Washington, May 4, 2011.

% \item K. M. Thyng and J. J. Riley, ``Modeling for Tidal Energy Analysis'', MoSSea Users Group, School of Oceanography, University of Washington, January 19, 2011.

% \item K. M. Thyng and J. J. Riley, ``Tidal Energy and Turbine Siting Metrics'', Mechanical Engineering Student Seminar, University of Washington, October 11, 2010.

\item K. M. Thyng and J. J. Riley, ``Idealized Headland Simulation for Tidal Hydrokinetic Turbine Siting Metrics'', OCEANS 2010 MTS/IEEE Seattle, September 21, 2010.

\item K. M. Thyng and J. J. Riley, ``Working Toward Numerical Simulations of Admiralty Inlet for Tidal Hydrokinetic Energy,'' \href{http://inore.org/}{4th Annual INORE Symposium, Dartmouth, UK, May 12, 2010}.

\item K. M. Thyng and J. J. Riley, ``Tidal Energy in the Puget Sound'', SIAM UW, April 21, 2009.

% \item K. M. Thyng and J. J. Riley, ``Tidal Energy in the Puget Sound'', SIAM UW, May 29, 2008.

\end{itemize}

\subsection*{Poster Presentations}

\begin{itemize}

\item K. M. Thyng, ``Perceptual colormaps in matplotlib with an application in oceanography,'' SciPy Conference 2015, Austin, TX, July 8, 2015.

\item K. M. Thyng and R. D. Hetland, ``Texas-Louisiana Shelf and Coast Connectivity,'' Gordon Research Conference: Coastal Ocean Modeling, University of New England, June 7--12, 2015.

\item T. Roc, K. M. Thyng, and S. W. Funke, ``Benchmarking Tidal Array Optimization: a Balance between Impacts \& Economics of the Bay of Fundy - Early Results,'' 5th International Conference on Ocean Energy, November 4--6, 2014, Halifax, Canada.

\item K. M. Thyng and R. D. Hetland, ``Cross-shelf transport and dispersion due to baroclinic instabilities,'' European Geosciences Union General Assembly 2014, April 27--May 2, Vienna, Austria.

\item K. M. Thyng and R. D. Hetland, ``Texas-Louisiana Cross-shelf Transport due to Submesoscale Eddies,'' Ocean Sciences Meeting, Honolulu, Hawaii, February 23--28, 2014.

\item J. Kuehl, K. M. Thyng, and P. Chapman, ``GISR Drift Card Program: Surface Transport Observation,'' Gulf of Mexico Oil Spill and Ecosystem Science Conference, Mobile, Alabama, January 26--29, 2014.

\item K. M. Thyng and R. D. Hetland, ``Texas-Louisiana Shelf Connectivity and Time Variability using Particle Tracking,'' Gulf of Mexico Oil Spill and Ecosystem Science Conference, Mobile, Alabama, January 26--29, 2014.

\item K. M. Thyng and T. Roc, ``Tidal current turbine power capture and impact in an idealised channel simulation,'' 10th European Wave and Tidal Energy Conference, Aalborg, Denmark, September 2--5, 2013.

\item K. M. Thyng, J. J. Riley, and M. Kawase, ``Vorticity Dynamics in Admiralty Inlet, Puget Sound,'' Gordon Research Conference: Coastal Ocean Circulation, University of New England, June 9--14, 2013.

\item \href{http://pong.tamu.edu/~kthyng/posters/pecs.pdf}{ROMS Turbulence Parameter Comparisons with Field Data}, The Physics of Estuaries and Coastal Seas (PECS) Symposium, 12--16 August 2012, New York City.

\item Nested ROMS Model of a Complex Estuarine Channel, Puget Sound, WA. Gordon Research Conference: Coastal Ocean Modeling,
  Mt. Holyoke College, South Hadley,  MA, 
  June 26--July 1, 2011.
  
\item \href{http://froude.me.washington.edu/presentations/GPSS051311.pdf}{Site Modeling for Tidal Turbines}. Graduate and Professional Student Senate Science and Policy Summit, University of Washington, May 13, 2011.

\item \href{http://froude.me.washington.edu/presentations/InorePoster2010/inore.pdf}{Numerical Modeling for Tidal
Hydrokinetic Turbine Siting}. \href{http://www.inore.org}{4th Annual INORE Symposium, Dartmouth, UK, May 9, 2010}.

\item \href{http://froude.me.washington.edu/presentations/inore_poster2009.pdf}{Estuary Modeling for Tidal Energy in Puget Sound, WA}. \href{http://www.inore.org}{3rd Annual INORE Symposium, Gent, Belgium, May 26, 2009}.

\end{itemize}

\subsection*{Grants}
\begin{itemize}
  \item Design of a Modern Web Interface to TGLO TABS Model and Data Products -- Phase 2, Texas General Land Office, September 1, 2015 -- August 31, 2017, \$186,988, PI: K. M. Thyng, co-PI: R. D. Hetland.
  \item Improving Oil Spill Predictions Near Shore and Across The Bay/Coastal Interface, Texas General Land Office, September 1, 2015 -- August 31, 2017, \$406,910, PI: B. R. Hodges, co-PIs: S. A. Socolofsky, K. M. Thyng.
  \item Improving Hydrodynamic Predictions of Surface Currents Near the Texas Coast Used for Rapid Oil Spill Response -- Phase 4, Texas General Land Office, September 1, 2015 -- August 31, 2017, \$376,560, PI: R. D. Hetland, co-PI: K. M. Thyng.
\end{itemize}

% \subsection*{Other Funding Received}

% \begin{itemize}
%   % \item 7th U.S. Harmful Algae Symposium travel award (\$500) based on statement (2013).
%   \item \href{http://inore.org/news/icis_travel_grants/}{International Collaboration Incentive Scheme grant}, a collaborative research grant through the International Network for Offshore Renewable Energy (INORE) (2011).
%   % \item Travel and conference funding for Gordon Research Conference: Ocean Modeling (2011).
%   % \item Travel funding from the Graduate School Fund for Excellence and Innovation (GSFEI), University of Washington (2010).
%   % \item Symposium and travel funding, International Network of Offshore Renewable Energy researchers (2010)
%   % \item Symposium and travel funding, International Network of Offshore Renewable Energy researchers (2009)
%   % \item Energy Summer School funding, U.K. Energy Research Council (2008).
% \end{itemize}

\subsection*{Other Conferences and Workshops Attended}

\begin{itemize}

\item MPOWIR Pattullo Conference for women in physical oceanography, Airlie Center, Warrenton, Virginia, October 6--9, 2013.

% \item The 12th International workshop on Multi-scale (Un)-structured mesh numerical Modeling for coastal, shelf, and global ocean dynamics (IMUM), University of Texas, Austin, September 16--19, 2013.

% \item Gordon Research Seminar and Gordon Research Conference: Coastal Ocean Circulation, University of New England, June 9--14, 2013.

% \item Association for the Sciences of Limnology and Oceanography (ASLO) 2013 Aquatic Sciences Meeting, February 17--22, 2013, New Orleans, Louisiana.

% \item Gulf Of Mexico: Oil Spill \& Ecosystem Science Conference, January 21--23, 2013, New Orleans, Louisiana.

\item Subsea Blowout Modeling Workshop, University of California, Berkeley, November 27--28, 2012.

% \item The Physics of Estuaries and Coastal Seas (PECS) Symposium, 12-16 August 2012, New York City.

% \item \href{http://www.grc.org/programs.aspx?year=2011&program=coastal}{Gordon Research Conference: Coastal Ocean Modeling},
  % Mt. Holyoke College, South Hadley,  MA, 
  % June 26--July 1, 2011.

% \item \href{http://www.oceans10mtsieeeseattle.org/}{OCEANS 2010 MTS/IEEE},
%   Seattle,
%   September 20--23, 2010.

% \item \href{http://www.inore.org}{4th Annual International Network of Offshore Renewable Energy researchers (INORE) Symposium},
%   Dartmouth, UK,
%   May 9--14, 2010.

% \item \href{http://www.agu.org/meetings/os10/}{2010 Ocean Sciences Meeting},
  % Oregon Convention Center,
  % February 22--26, 2010.

% \item \href{http://www.oce.uri.edu/ecm11/index.html}{Eleventh International Conference on Estuarine and Coastal Modeling},
  % Seattle, WA,
  % November 4--6, 2009.

% \item \href{http://www.inore.org}{3rd Annual International Network of Offshore Renewable Energy researchers (INORE) Symposium},
%   Gent, Belgium,
%   May 24--28, 2009.

\item \href{http://www.ukerc.ac.uk/support/tiki-index.php?page=0608UKERCSummerSchool}{U.K. Energy Research Council (UKERC) Energy Summer School 2008},
  Roehampton, London, UK,
  June 23--27, 2008.

\end{itemize}

\section*{Teaching}

\subsection*{Texas A\&M University, Department of Oceanography}

\begin{itemize}

\item Python for Geoscientists (OCNG 489/689), Spring 2016, undergrad/graduate class.
\item Introduction to Oceanography (OCNG 251), Fall 2015, undergraduate class (68 students).
\item Introduction to Oceanography (OCNG 251), Spring 2015, undergraduate class (46 students).
% \item Guest Lecturer: Introduction to Oceanography (OCNG 251), Spring 2014, undergraduate class.
% \item Guest Lecturer: Python for Geoscientists (OCNG 689), Fall 2013, graduate class.
% \item Guest Lecturer: Introduction to Physical Oceanography (OCNG 410), Spring 2013, undergraduate class.

\end{itemize}

\subsection*{University of Washington, Department of Mechanical Engineering}

\begin{itemize}

\item Teaching Assistant,
  \href{http://www.washington.edu/students/crscat/meche.html#me323}{Thermodynamics, ME 323},
  \href{http://www.me.washington.edu/research/faculty/malte/}{Dr. Philip Malte},
  Fall 2007.

\end{itemize}

% \subsection*{Workshops}

% \begin{itemize}
%   \item Teaching Methods, Center for Teaching Excellence, Texas A\&M University (2013).
% \end{itemize}

% \subsection*{Tutoring}

% \begin{itemize}
% 	\item Trigonometry and physics, November - December 2007.
% 	\item Algebra, geometry and trigonometry, January - March 2007.
% 	\item AP Calculus BC exam, March - May 2007.
% 	\item Calculus, June - July  2007.
% \end{itemize}

\section*{Honors \& Awards}

\begin{itemize}

\item SciPy John Hunter Excellence in Plotting Competition 2015: Honorable mention.

\item SciPy John Hunter Excellence in Plotting Competition 2014: 3rd place.

\item NASA MPOWIR (Mentoring Physical Oceanography Women to Increase Retention) selected speaker, 2014.

\item SciPy John Hunter Excellence in Plotting Competition 2013: 2nd place.

\item Outstanding Female Award, Mechanical Engineering. Society of Women Engineers, University of Washington. January 25, 2012.

\item \href{http://www.inore.org}{Best Symposium Poster}, INORE Symposium, 2010.

% \item \href{https://www.whitman.edu/content/catalog/financial-aid}{Walter Brattain Scholarship}, 2001-2005.

%\item Holy Mother Marie Rose Scholarship, 1997-2001.
%
%\item Valedictorian, 2001

\end{itemize}

\section*{Service}
\begin{itemize}
  % \item Program Committee, SciPy Conference 2015.
  \item Associate Chair for Gordon Research Seminar on Coastal Ocean Modeling (2015).
  \item Co-chair for tutorials, SciPy Conference 2014 and 2015.
  \item Program Committee, SciPy Conference 2014 and 2015.
  \item Organizer and fund raiser for diversity event at SciPy conference, 2013 and 2014.
  % \item Diversity Committee: co-organizer and fund raiser for event, SciPy Conference 2014.
  % % \item Co-chair for tutorials, SciPy Conference 2014.
  % \item Organizer and fund raiser for Women in Scientific Computing event, June 27, 2013.
	\item \href{http://www.amath.washington.edu/~siamuw/math-fair.html}{Math and Science Fair}, Lockwood Elementary School, December 15, 2010 and March 15, 2011; Emerson Elementary School, June 8, 2010.
	% \item \href{http://www.amath.washington.edu/~siamuw/math-fair.html}{Math and Science Fair}, Lockwood Elementary School, December 15, 2010.
	% \item Math and Science Fair, Emerson Elementary School, June 8, 2010.
	\item An Introduction to Tidal Energy Research in the Puget Sound. UW Robinson School Summer Challenge, July 19, 2010.
	% \item Volunteer at \href{http://www.globalmarinerenewable.com/}{3rd Annual Global Marine Renewable Energy Conference},  Bell Harbor International Convention Center,  April 14--15, 2010.
  \item Referee: Journal of Geophysical Research - Oceans; Maryland Sea Grant, Estuaries \& Coasts; EuroSciPy; Journal of Marine Science and Engineering; NSF: Division of Ocean Sciences; Ocean Dynamics; Packt Publishing; International Conference on Ocean, Offshore and Arctic Engineering (OMAE); National Oceanic and Atmospheric Administration; European Wave and Tidal Energy Conference
\end{itemize}


\subsection*{Professional Activities}

\begin{itemize}

\item Member, \href{http://www.agu.org/}{American Geophysical Union} (2010--Present); \href{http://www.siam.org/}{Society for Industrial and Applied Mathematics}, (2008--2013); \href{http://www.aslo.org/}{Association for the Sciences of Limnology and Oceanography} (2012--2013); \href{http://www.ieee.org/}{Institute of Electrical and Electronics Engineers} (2010--2012).

%\item Referee for:
%
%  \begin{itemize}
%  \item \textit{Economic Inquiry} % 2008 (1)
%  \item \textit{Journal of Business and Economic Statistics} % 2009 (1)
%  \item \textit{Journal of Econometrics} % 2009 (1)
%  \item \textit{Scandinavian Journal of Statistics} % 2009 (1)
%  \end{itemize}

\end{itemize}

\section*{Skills}
\begin{itemize}
	\item Extensively used SUNTANS and ROMS ocean modeling codes.
	\item Proficiency in Python, FORTRAN, C, MatLab, and \LaTeX.
	\item Linux system administration.
	\item Experience using a cluster and parallel computing.
	\item Experience with HTML and CSS.
\end{itemize}

% \section*{Other Interests}
% \begin{itemize}
% 	\item Hiking
% 	\item Mountaineering
% 	\item Road and trail running
% \end{itemize}

% \bigskip

% Footer
\begin{center}
  \begin{small}
    Last updated: \today
  \end{small}
\end{center}

\end{document}
